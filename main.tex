%!TEX TS-program = XeLaTeX
%!TEX encoding = UTF-8 Unicode
\documentclass[UTF8, 12pt, a4paper]{ctexart}
\usepackage{amsmath, amssymb, graphicx, geometry, enumitem, booktabs}
\usepackage[
  unicode,
  colorlinks=true,
  bookmarksopen=true,
  bookmarksnumbered=true,
  hidelinks, 
  pdftitle={sx_230422}, 
  pdfauthor={梁樱祖}, 
  pdfkeywords={抛物线,试题}]{hyperref}
\setlength{\parindent}{2em}
\setlength{\parskip}{0em}
\renewcommand{\labelenumi}{Q\theenumi.}
\newcommand \qref[1]{Q\ref{#1}}
\geometry{left=3cm, right=3cm, top=3cm, bottom=3.25cm}
\title{sx_230422}
\author{梁樱祖}
\date{2023-04-22}
\pagestyle{plain}

\begin{document}
已知抛物线$\Gamma:y^2=4x$,$R\left(2,0 \right)$,直线$l:y=k_1\left(x-1 \right)$交$\Gamma$于$A\left(x_1,y_1 \right)$、$B\left(x_2,y_2 \right)$两点($y_1>0$),直线$AR$、$BR$分别交$\Gamma$于另一点$C\left(x_3,y_3 \right)$、$D\left(x_4,y_4 \right)$,$AB$交$CD$于$T$,$AD$交$BC$于$S$.
\begin{enumerate}[itemsep=0pt, partopsep=0pt, parsep=\parskip, topsep=0pt]
\item 设$CD$斜率为$k_2$,求$k_1:k_2$.
\item $CD$过定点$E$,求$E$点坐标.
\item $AD$、$BC$分别与$l':x=2$交于$M$、$N$,求证:$\left|RM \right|=\left|RN \right|$.
\item 求证:$S$、$T$在同一定直线$l_0$上.
\item 当$\triangle ABC$面积最小时,求$x_1$.
\item $AC$交$ST$于$G$,求证:$\left|GA \right|\left|RC \right|=\left|RA \right|\left|GC \right|$.
\item 当$\triangle SRT$面积最小时,求$k_1$.
\end{enumerate}

\begin{enumerate}
\item \label{q1}$\because A$、$B$、$F$三点共线,
\begin{equation}
    \therefore \dfrac{y_1}{x_1-1}=\dfrac{y_2}{x_2-1}\ \Rightarrow \ y_1y_2=-4\label{eq1}
\end{equation}
同理,由$A$、$C$、$R$和$B$、$D$、$R$三点共线,
\begin{equation}
    \Rightarrow\ y_1y_3=y_2y_4=-8\label{eq2}
\end{equation}
又因为
\begin{equation}
k_1=\dfrac{y_1-y_2}{x_1-x_2}=\dfrac{4\left(y_1-y_2 \right)}{y_1^2-y_2^2}=\dfrac{4}{y_1+y_2}\label{eq3}
\end{equation}
$\therefore$ 同理,
\[k_2=\dfrac{4}{y_3+y_4}=\dfrac{4y_1y_2}{y_1y_3\cdot y_2+y_2y_4\cdot y_1}=\dfrac{1}{2}\cdot\dfrac{4}{y_1+y_2}=\dfrac{1}{2}k_1\]
$\therefore\ k_1:k_2=2$.
\item \label{q2}由\eqref{eq1}、\eqref{eq2}可得
\begin{equation}
y_3y_4=-16\label{eq4}
\end{equation}
$\because C\left(x_3,y_3 \right)$、$D\left(x_4,y_4 \right)$
\begin{equation}
\Rightarrow CD:4x-\left(y_3+y_4 \right)y+y_3y_4=0\label{eq5}
\end{equation}
$\therefore$由\eqref{eq4},可得
\begin{equation}
CD:4x-\left(y_3+y_4 \right)y-16=0\label{eq6}
\end{equation}
$\therefore CD$过定点$E\left(4,0 \right)$.
\item \label{q3}由\eqref{eq5},同理可得$AD:4x-\left(y_1+y_4 \right)y+y_1y_4=0$\par
又由\eqref{eq1}、\eqref{eq2},可得
\begin{equation}
AD:y_2x+3y-2y_1=0\label{eq7}
\end{equation}
同理,
\begin{equation}
BC:y_1x+3y-2y_2=0\label{eq8}
\end{equation}
$\therefore$\eqref{eq7}、\eqref{eq8}分别与$l'$联立可得
\[y_M=\dfrac{2}{3}(y_1-y_2),\ y_N=\dfrac{2}{3}(y_2-y_1)\]
$\therefore\dfrac{\left|RM\right|}{\left|RN\right|}=\left|\dfrac{y_M}{y_N}\right|=1$,即$\left|RM \right|=\left|RN \right|$.
\item \label{q4}联立\eqref{eq7}、\eqref{eq8},可得
\begin{equation}
    \begin{cases}
    x_S=\dfrac{2y_1-2y_2}{y_2-y_1}=-2\\[2mm]
    y_S=\dfrac{2}{3}\left(y_1+y_2\right)
    \end{cases}\label{eq9}
\end{equation}
由$l$、\qref{q1}、\eqref{eq6},可得
\begin{equation}
    CD:k_1x-2y-4k_1=0\label{eq10}
\end{equation}
联立$l$、\eqref{eq10},可得
\begin{equation}
    \begin{cases}
    x_T=-2\\
    y_T=-3k_1\label{eq11}
    \end{cases}
\end{equation}
$\therefore S$、$T$在同一定直线$l_0:x=-2$上.
\item \label{q5}设$\Gamma$的焦点为$F$,由\eqref{eq1}、\eqref{eq2},则
\begin{align*}
    S_{\triangle ABC}&=S_{\triangle AFC}+S_{\triangle BFC}\\
    &=\dfrac{y_1-y_2}{y_1}\cdot S_{\triangle AFC}\\
    &=\dfrac{y_1-y_2}{y_1}\cdot\dfrac{y_1-y_3}{2}\\
    &=\dfrac{1}{2}\left(y_1-y_2-y_3+\dfrac{y_2y_3}{y_1}\right)\\
    &=\dfrac{1}{2}\left(y_1+\dfrac{12}{y_1}+\dfrac{32}{y_1^3}\right)\\
    \Rightarrow S'_{\triangle ABC}&=\dfrac{1}{2y_1^4}\left(y_1^4-12y_1^2-96\right)\\
    &=\dfrac{8}{y_1^4}\left(x_1^2-3x_1-6\right)
\end{align*}
$\therefore$令$S'_{\triangle ABC}=0$,则$x_1=\dfrac{3+\sqrt{33}}{2}$,\par
$\therefore$易知$S_{\triangle ABC}$在$x_1\in\left(0,\dfrac{3+\sqrt{33}}{2} \right)\searrow$,在$x_1\in\left(\dfrac{3+\sqrt{33}}{2},+\infty \right)\nearrow$\par
即$x_1=\dfrac{3+\sqrt{33}}{2}$时,$S_{\triangle ABC}$取最小值.
\item \label{q6}由\qref{q4},可得
\begin{align*}
&\left|GA \right|\left|RC \right|=\left|RA \right|\left|GC \right|\\
\Leftrightarrow\quad&\dfrac{\left|GA \right|}{\left|GC \right|}=\dfrac{\left|RA \right|}{\left|RC \right|}\\
\Leftrightarrow\quad&\left|\dfrac{x_1+2}{x_3+2} \right|=\left|\dfrac{x_1-2}{x_3-2} \right|\\
\Leftrightarrow\quad&\left|x_1x_3+2x_1-2x_3-4 \right|=\left|x_1x_3-2x_1+2x_3-4 \right|
\end{align*}
又由\eqref{eq2},得
\begin{equation}
    x_1x_3=\frac{y_1^2}{4}\cdot\frac{y_3^2}{4}=\frac{\left(y_1y_3 \right)^2}{16}=4
\end{equation}
$\therefore$原等式成立.
\item \label{q7}由\eqref{eq3}、\eqref{eq9}、\eqref{eq11},可得
\[S\left(-2,\frac{8}{3k_1} \right),\ T\left(-2,-3k_1 \right)\]
设$ST$交$x$轴于$H$,可得
\begin{align*}
S_{\triangle SRT}&=\frac{1}{2}\cdot\left|HR \right|\cdot\left|y_S-y_T \right|\\
&=2\left|\frac{8}{3k_1}+3k_1 \right|\\
&=2\left(\frac{8}{3\left|k_1\right|}+3\left|k_1\right| \right)\\
&\ge 2\cdot2\sqrt{\frac{8}{3\left|k_1\right|}\cdot3\left|k_1\right|} \\
&=8\sqrt{2}
\end{align*}
$\therefore$当且仅当$\dfrac{8}{3\left|k_1\right|}=3\left|k_1\right|$,即$k_1=\pm\dfrac{2\sqrt{2}}{3}$时,$S_{\triangle SRT}$取最小值$8\sqrt{2}$.
\end{enumerate}
\end{document}
